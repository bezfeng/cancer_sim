\documentclass[11pt,a4paper]{article}
\usepackage[utf8]{inputenc}
\usepackage{helvet}


\usepackage{authblk}
\usepackage{lineno}
\usepackage{caption}
\usepackage{cite}
\usepackage{url}
\usepackage[utf8]{inputenc}
\usepackage{helvet}
\usepackage{todonotes}
\usepackage{amsmath}
%\usepackage{amthrsfs}
\usepackage{dsfont}
\usepackage{mathrsfs}

\usepackage{amsfonts}
\usepackage{hyperref}
\usepackage{amsthm}
\usepackage{graphicx}
\usepackage{subfigure}
\usepackage{xcolor}
\usepackage{authblk}
%\renewcommand{\qed}{\hfill\small{$\square$}\normalsize}
%alsize}


\DeclareFixedFont{\Acknowledgment}{OT1}{cmr}{bx}{n}{14pt}
\textwidth 150mm \textheight 200mm \hoffset -1.2cm \voffset -0.5cm
\linespread{1.1}


\begin{document}
\title{PICKA Python Interactive Cancer Knowledge Analyser}

\author[1]{Luka Opasic}
\author[1]{Carsten Fortmann-Grotte}

\affil[1]{Max Planck Instutute for Evolutionary Biology, Ploen, Germany}

\date{}
\maketitle


\section{Summary}


Guidance:
Authors include in the paper some sentences that would explain the software functionality and domain of use to a non-specialist reader. Your submission should probably be somewhere between 250-1000 words. A summary describing the high-level functionality and purpose of the software for a diverse, non-specialist audience.
A clear statement of need that illustrates the purpose of the software. Mentions (if applicable) of any ongoing research projects using the software or recent scholarly publications enabled by it






\end{document}
